% !TEX TS-program = pdflatex
% !TEX encoding = UTF-8 Unicode

% This is a simple template for a LaTeX document using the "article" class.
% See "book", "report", "letter" for other types of document.

\documentclass[11pt]{article} % use larger type; default would be 10pt

\usepackage[utf8]{inputenc} % set input encoding (not needed with XeLaTeX)


%%% PAGE DIMENSIONS
\usepackage{geometry} % to change the page dimensions
\geometry{a4paper} % or letterpaper (US) or a5paper or....


\usepackage{graphicx} % support the \includegraphics command and options

% \usepackage[parfill]{parskip} % Activate to begin paragraphs with an empty line rather than an indent

%%% PACKAGES
\usepackage{booktabs} % for much better looking tables
\usepackage{array} % for better arrays (eg matrices) in maths
\usepackage{paralist} % very flexible & customisable lists (eg. enumerate/itemize, etc.)
\usepackage{verbatim} % adds environment for commenting out blocks of text & for better verbatim
\usepackage{subfig} % make it possible to include more than one captioned figure/table in a single float
\usepackage{cite}
% These packages are all incorporated in the memoir class to one degree or another...

%%% HEADERS & FOOTERS
\usepackage{fancyhdr} % This should be set AFTER setting up the page geometry
\pagestyle{fancy} % options: empty , plain , fancy
\renewcommand{\headrulewidth}{0pt} % customise the layout...
\lhead{}\chead{}\rhead{}
\lfoot{}\cfoot{\thepage}\rfoot{}

%%% SECTION TITLE APPEARANCE
\usepackage{sectsty}
\allsectionsfont{\sffamily\mdseries\upshape} % (See the fntguide.pdf for font help)
% (This matches ConTeXt defaults)

%%% ToC (table of contents) APPEARANCE
\usepackage[nottoc,notlof,notlot]{tocbibind} % Put the bibliography in the ToC
\usepackage[titles,subfigure]{tocloft} % Alter the style of the Table of Contents
\renewcommand{\cftsecfont}{\rmfamily\mdseries\upshape}
\renewcommand{\cftsecpagefont}{\rmfamily\mdseries\upshape} % No bold!



\title{Extraction of Novel Biological Priors from Skeletal Neuronal Images for the Classification of (EM) Connectomics Data}
\author{Elizabeth Morgan, Katie Link, Sandra Weiss, Eileen Xu}
%\date{} % Activate to display a given date or no date (if empty),
         % otherwise the current date is printed 

\begin{document}
\maketitle

\section{Introduction}
Producing a connectome from electron microscopy (EM) images provides an incredibly informative graph of connections on the individual neuron level. However, the enormous amount of data, the time it takes to manually annotate it, and the many errors automated parsing software produces are major challenges facing the continued growth of the burgeoning field of connectomics. [1][2] In this proposal, we present a project that attempts to use existing skeletal and EM data for analysis and classification of new biological priors, such as dendrite branching angles. Building off work presented in Roncal et al.'s 2014 Society for Neuroscience poster, [3] this project will be accomplished using recently developed software such as the TREES software for MATLAB [4] and analyzing neuronal skeletal data from NeuroMorpho.org [5] as well as ground truth data for comparison. These priors will be meaningful to improve accuracy and speed of parsing neurons in EM data after being implemented in future automatic annotation and proofreading software by computer vision scientists. \\


\section{Project Outline}
\subsection{Methods}
Structural trees are available publicly on NeuroMorpho.org [5] and will provide a starting point for data collection. We will try to distinguish between interneurons and pyramidal neurons in the neocortex of mice using the TREES program. [4] We will compare the priors found by analyzing the neurons from NeuroMorpho.org to the gold standard for EM data on the neocortex. This way we can test whether these priors are valid. Then we will compare these types of neurons between different parts of the brain, for instance, between the neocortex and hippocampus. 

\subsection{Schedule}

\textbf{Monday, January 11th}\\
Each team member will have MATLAB and the TREES toolbox [4] up and running with some sample data from NeuroMorpho.org [5]. There will be a possible meeting on Sunday, January 10th to go over any remaining problems.\\
\\
\textbf{Wednesday, January 13th} \\
After working with the NeuroMorpho.org and other data for a few days now, we will have identified at least four biological priors learned from the data that we will intend to pursue. We will delegate a prior to each team member by this time.\\
\\
\textbf{Friday, January 15th}\\
By this date, we should have preliminary results for most of the skeletal data, which will be compiled in the progress report. Any preliminary successes and problems will be included. We will also begin comparing the these priors with gold standard data.\\
\\
\textbf{Monday, January 18th} \\
Over the weekend, we will work through any remaining problems individually and as a group. An outline of our final presentation and report will be collaborated on over the weekend and produced by Monday.\\
\\
\textbf{Wednesday, January 20th}\\
By the end of this date, the final poster and presentation will be finished as a collaborative effort by all team members.

\subsection{Allocation of Tasks}
All team members will try to distinguish between the inter-neurons and pyramidal cells in the mouse cortex by applying the templates on NeuroMorpho.org [5] to the TREES toolbox in MATLAB and find priors, which will then be compared to the gold standard. Afterwards, each member will be allocated a prior to compare cells between different regions of the brain. The entire group will come together frequently to solve any emerging problems or make sure everyone is on task.

\section{References}
% couldn't figure out how to do the fancy LaTeX bibliography D:
1. Jurrus, Elizabeth, et al. "Detection of neuron membranes in electron microscopy images using a serial neural network architecture." Medical image analysis 14.6 (2010): 770-783.\\
\\
2. Plaza, Stephen M. "Focused Proofreading: Efficiently Extracting Connectomes from Segmented EM Images." arXiv preprint arXiv:1409.1199 (2014).\\
\\
3. Roncal, William Gray, et al. "A Semantic Framework to Guide Computer Vision in (EM) Connectomics." Society for Neuroscience. Washington, DC. 11-15 Nov. 2014.\\
\\
4. Cuntz, Hermann, et al. "The TREES toolbox—probing the basis of axonal and dendritic branching." Neuroinformatics 9.1 (2011): 91-96.\\
\\
5. Ascoli, Giorgio A., Duncan E. Donohue, and Maryam Halavi. "NeuroMorpho.Org: a central resource for neuronal morphologies." The Journal of Neuroscience 27.35 (2007): 9247-9251.\\


\end{document}